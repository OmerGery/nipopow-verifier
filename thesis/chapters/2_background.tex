\chapter{Background}

Relevant technologies

\section{Primitives}

Describe primitives

\section{Bitcoin}

Describe Bitcoin blockchain

\section{Ethereum}

Describe Ethereum blockchain

\subsection{Solidity}

Describe the use of solidity language

\subsection{Smart contracts}

Describe the use of smart contracts

\subsection{Ethereum Virtual Machine}

The Ethereum Virtual Machine (EVM) is a sandboxed virtual stack
embedded within each full Ethereum node, responsible for executing
contract bytecode. Contracts are typically written in higher level
languages, like Solidity, then compiled to EVM bytecode.

This means that the machine code is completely isolated from the
network, filesystem or any processes of the host computer. Every node
in the Ethereum network runs an EVM instance which allows them to
agree on executing the same instructions. The EVM is Turing complete,
which refers to a system capable of performing any logical step of a
computational function. JavaScript, the programming language which
powers the worldwide web, widely uses Turing completeness.

Ethereum Virtual Machines have been successfully implemented in
various programming languages including C++, Java, JavaScript, Python,
Ruby, and many others.

The EVM is essential to the Ethereum Protocol and is instrumental to
the consensus engine of the Ethereum system. It allows anyone to
execute code in a trustless ecosystem in which the outcome of an
execution can be guaranteed and is fully deterministic (i.e.)
executing smart contracts.

\subsection{Non-Interactive Proofs Of Proof Of Work}

Describe the rationale behind NIPoPoWs, what they provide

\subsection{Prefix Proofs}

Describe prefix proofs

\subsection{Suffix Proofs}

Describe suffix proofs

\subsection{Infix Proofs}

Describe infix proofs

\section{Forks}

Soft, hard and velvet fork

\subsection{Difficulty}

Describe constant and non-constant difficulty
