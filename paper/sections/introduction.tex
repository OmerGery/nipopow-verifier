\section{Introduction}

Blockchain interoperability~\cite{dionyziz} is the ability of distinct
blockchains to communicate.  This \emph{crosschain}~\cite{pow-sidechains,
pos-sidechains,burn,crosschain-sok, gtklocker} communication enables useful
features across blockchains such as the transfer of asset from one chain to
another (one-way peg) and back (one-way peg)~\cite{pow-sidechains}. To date,
there is no commonly accepted decentralized protocol that enables cross-chain
transactions. Currently, crosschain operations are only available to the users
via third-party applications, such as multi-currency wallets. However, this
treatment is opposing to the nature of decentralized currencies.

\noindent

In general, crosschain-enabled blockchains A, B support the following
operations:

\begin{itemize}
\item Crosschain trading: A user with deposits in blockchain A, makes a
    payment to a user in blockchain B.
\item Crosschain fund transfer: A user transfers her funds from blockchain
    A to blockchain B. After the transfer, these funds no longer exist in
    blockchain A. The user can later decide to transfer any portion of the
    original amount to the blockchain of origin.
\end{itemize}


\noindent

In order to perform crosschain operations, there must be a mechanism to allow
for users of blockchain A to discover events that occur in chain B, such that a
transaction occurred. A trivial manner to perform such an audit is to
participate as a full node in multiple chains. This approach, however, is
impractical because a sizeable amount of storage is needed to host entire
chains as they grow with time. As of July 2020, Bitcoin~\cite{nakamoto} chain
spans roughly 245 GB, and Ethereum~\cite{wood, buterin} has exceeded 350
GB\footnote{The size of the Bitcoin chain was derived from
    https://www.statista.com/statistics/647523/worldwide-bitcoin-blockchain-size/,
    and the size of the Ethereum chain by
https://etherscan.io/chartsync/chaindefaults}. Naturally, not all users are
able to accommodate this size of data, especially if portable devices are used,
such as mobile phones.

One early solution to compress the extensive size of blockchain is addressed by
Nakamoto~\cite{nakamoto} with the Simplified Payment Verification (SPV)
protocol. In SPV, only the headers of blocks are stored, saving a
considerable amount of storage.  However, even with this protocol, the process
of downloading and validating all block headers leads to unpleasant user
experience. In Ethereum, for instance, headers sum up to approximately 5.1
GB\footnote{Calculated as the number of blocks (10,050,219) times the size of
header (508 bytes). Statistics by https://etherscan.io/} of data. A mobile
client needs several minutes, even hours, to fetch all information needed in
order to function as an SPV client.

Towards the goal of delivering more practical solutions for blockchain
transaction verification, a new generation of \emph{superlight}
clients~\cite{popow,nipopows,compactsuperblocks, flyclient} emerged. In these
protocols, cryptographic proofs are generated, that prove the occurrence of
events inside a blockchain. Better performance is achieved due to the
considerably smaller size of proofs compared to the amount of data needed in
SPV. By utilizing superlight client protocols, a compressed proof for an event
in chain A is constructed and dispatched to chain B. Under the condition that
chain B supports smart contracts, the proof is then verified automatically and
transparently \emph{on-chain}. This communication is realized without the
intervention of third-party applications. An interesting application of such a
protocol is the communication between Bitcoin and Ethereum.

\noindent

\textbf{Related Work.} We use Non-Interactive Proofs of Proof of Work
(NIPoPoWs)~\cite{nipopows, pow-sidechains} as the fundamental building block
of our solution. This cryptographic primitive is \emph{provably secure} and
provides \emph{succinct proofs} regarding the existence of an arbitrary event
in a chain. Contrary to the linear growth rate of the underlying blockchain,
NIPoPoWs span polylogarithmic size of blocks.

Christoglou~\cite{gglou} provided a Solidity smart contract which is the first
implementation of crosschain events verification based on NIPoPoWs, where
proofs are submitted and checked for their validity. This solution, however, is
impractical due to extensive usage of resources, widely exceeding the Ethereum
block gas limit.

Other attempts have been done to address the verification of Bitcoin
transactions to the Ethereum blockchain, most notably BTC
Relay~\cite{btcrelay}.

\noindent

\textbf{Our contribution.} Notably, no practical implementation for superlight
clients exists to date. In this paper, we focus on constructing a practical
client for NIPoPoWs. For the implementation of our client, we refine the
NIPoPoW protocol based on a series of keen observations. These refinements
allow us to leverage useful techniques that construct a practical solution for
proof verification. We believe that this achievement is a decisive step towards
the establishment of NIPoPoWs to the application end, therefore it is a
significant progress in order to provide a widely accepted protocol that
enables crosschain transactions. A summary of our contributions in this paper
is as follows:
\begin{enumerate}
\item We developed the first decentralized client that securely verifies
crosschain events and is practical. Our client establishes a trustless and
efficient solution to the interoperability problem. We implement our client
in Solidity, and we verify Bitcoin events to the Ethereum blockchain. The
security assumptions we make are no others than
SPV~\cite{eclipse, eclipse-ethereum}.
\item We present a novel pattern which we term \emph{hash-and-resubmit}. Our
pattern significantly improves performance of Ethereum smart
contracts~\cite{wood, buterin} in terms of gas consumption by utilizing
\emph{calldata} space of Ethereum blockchain to eliminate high-cost storage
operations.
\item We design an \emph{optimistic} schema which we incorporate
in the design of our client, that replaces \emph{non-optimistic} operations.
This design achieves the improvement of smart contracts' performance by
enabling multiple phases of interactions.
\item We demonstrate via application that the NIPoPoW protocol is practical,
making the cryptographic primitive the first provably secure construction of
succinct proofs that is efficient to implement.
\item Cryptoeconomics. Create the section first.

\end{enumerate}

Our implementation meets the following requirements:
\begin{enumerate}
\item Security: The client implements a provably secure protocol.
\item Decentralization: The client is not dependent on third-party applications
and operates in a transparent, decentralized manner.
\item Efficiency: The client comply with all environmental constraints, i.e.\
block gas limit and calldata size limit of Ethereum blockchain.
\end{enumerate}

We selected Bitcoin as source blockchain as it the most used cryptocurrency,
and enabling crosschain transactions in Bitcoin is beneficial to the vast
majority of blockchain community. We selected Ethereum as the target blockchain
because, besides its popularity, it supports smart contracts, which is a
requirement in order to perform on-chain verification.
\noindent

\textbf{Structure.} In Section 2 we describe the blockchain technologies that
are relevant to our work. In Section 3 we put forth the
\emph{hash-and-resubmit} pattern. We demonstrate the improved performance of
smart contracts using the pattern, and how it is incorporated in our superlight
client. In Section 4 we show ... and.  Finally, in Section 5, we discuss ...
