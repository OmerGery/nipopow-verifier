\begin{abstract}

Superlight clients enable the verification of proof-of-work-based blockchains
by checking only a small representative number of block headers instead of all
the block headers as done in SPV. Such clients can be embedded within other
blockchains by implementing them as smart contracts, allowing for cross-chain
verification. One such interesting instance is the consumption of Bitcoin data
within Ethereum by implementing a Bitcoin superlight client in Solidity. While
such theoretical constructions have demonstrated security and efficiency, no
practical implementation exists. In this work, we put forth the first
practical Solidity implementation of a superlight client which implements the
NIPoPoW superblocks protocol. Contrary to previous work, our Solidity smart
contract achieves sufficient gas-efficiency to allow a proof and counter-proof
to fit within the gas limit of a block, making it practical. We provide
extensive experimental measurements for gas. The optimizations that enable
gas-efficiency heavily leverage a novel technique which we term
hash-and-resubmit, which almost completely eliminates persistent storage
requirements, the most expensive operation of smart contracts in terms of gas.
Instead, the contract asks contesters to resubmit data and checks their
veracity by hashing it. Other optimizations include off-chain manipulation of
proofs in order to remove expensive look-up structures, and the usage of an
optimistic schema. We show that such techniques can be used to bring down gas
costs significantly and may have applications to other contracts. Lastly, our
implementation allows us to calculate concrete cryptoeconomic parameters for
the superblocks NIPoPoWs protocol and in particular to make recommendations
about the monetary value of the collateral parameters. We provide such
parameter recommendations over a variety of liveness settings.

\end{abstract}
