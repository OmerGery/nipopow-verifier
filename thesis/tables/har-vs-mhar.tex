\newcommand{\mydata}{\data}

\begin{table}[h]
\centering
\begin{tabular}{|c|c|c|}
\hline
\textbf{\begin{tabular}[c]{@{}c@{}}phase per\\variance\end{tabular}} &
\textbf{\begin{tabular}[c]{@{}c@{}}plain hash\\and resubmit\end{tabular}} &
\textbf{\begin{tabular}[c]{@{}c@{}}merkle hash\\ and resubmit\end{tabular}} \\ \hline
\textbf{hash} &
\textsf{H}($\mydata$) &
\begin{tabular}[c]{@{}c@{}}
    \textsf{H}($\mydata_{elem}$) $\times\ |\mydata|$ \\ \textsf{H}(digest)
$\times\ (|\mydata|-1)$

\end{tabular} \\ \hline
\textbf{resubmit} &
\textsf{load}($\mydata$) + \textsf{H}($\mydata$) &
\begin{tabular}[c]{@{}c@{}}
    \textsf{load}($\mydata[m{:}n])$ + \\
    \textsf{load}($siblings$) + \\
    \textsf{H}($\mydata[m{:}n])$ + \\
    \textsf{H}($digest$)$\times |siblings|$
\end{tabular} \\ \hline
\end{tabular}

\caption{Summary of operations for \emph{hash-and-resubmit} pattern variations.
$\mydata$ is the product of on-chain operations and $\mydata_{elem}$ is an
element of $\mydata$. \textsf{H} is a hash function, such as \textsf{sha256}
or \textsf{keccak}, $digest$ is the product of \textsf{H}(.) and $siblings$ are
the siblings of the Merkle Tree constructed for $\mydata$.
}

\label{tab:har-vs-mhar}
\end{table}
