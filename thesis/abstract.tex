\begin{abstract}

During the last years, significant effort has been put into enabling blockchain
interoperability, and several protocols have been proposed towards establishing
crosschain communication. Most notably, superlight clients, a new generation of
verifiers, have emerged. These clients demand only a poly-logarithmic number of
block headers in order to verify transactions, rather than the entire span of
the underlying chain. Albeit these constructions have been established
theoretically, no practical implementation exists to date. In this paper, we
focus on Non-Interactive Proofs of Proof of Work (NIPoPoWs), a probabilistic
structure based on superblocks that is provably secure and provides succinct
proofs of events in a blockchain. In particular, we discuss a gas-efficient
implementation for the verification of NIPoPoWs in Solidity, enabling
crosschain events from Bitcoin to Ethereum. We explore patterns and techniques
that considerably reduce gas consumption, and may have applications to other
smart contracts. We introduce a pattern that we term "hash-and-resubmit" that
eliminates persistent storage almost entirely, leading to significant increase
of performance. Furthermore, we alleviate the burden of expensive on-chain
operations, which we transfer off-chain, and we make use of an optimistic
schema that replaces functionalities of linear complexity with constant
operations. Lastly, we make a cryptoeconomic analysis, and set concrete values
regarding the cost that comes with using our client. Our implementation in
Solidity is accompanied by thorough unit tests. We display the performance gain
of our solution compared to previous work, and we mitigate security issues we
encountered such as premining.

\end{abstract}
