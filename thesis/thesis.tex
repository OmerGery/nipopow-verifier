\documentclass{article}

\usepackage[utf8]{inputenc}

\title{A gas-efficient superlight Bitcoin client in Solidity}
\begin{document}

  \maketitle

  \begin{abstract}
    Place abstract here
  \end{abstract}

  \pagebreak

  \section{Introduction}

  \subsection{Motivation}

  Digital coins are peer-to-peer currencies based on applied
  cryptography for the validation of transactions. Most of them are
  based on blockchain(ref), a form of decentralized database. In this
  database, a public ledger is deployed, which is stored and updated by
  thousands of users in absence of supervision from public authorities.

  In 2008, Bitcoin(ref), the first ever successful decentralized digital
  coin, was invented by an unknown person or group of people using the
  name Satoshi Nakamoto. A year after, the bitcoin network started,
  quickly followed by several other digital coins, which in the
  cryptocurrency folklore are known as altcoins. Usually, altcoins are
  based on innovative features missing from the cryptocurrency market,
  and they are either accepted or rejected by the community. Popular
  altcoins are Ethereum(ref), which is the first to provide smart
  contracts, Ripple(ref) which provides real-time payment settlements
  and Litecoin(ref) which enables near-zero cost payments.

  Over the last decade, cryptocurrencies gained attention from the
  public as an increased number of users are accepting and adapting to
  decentralized transactions. Specifically, in 2017, the popularity of
  cryptocurrencies rapidly grew, resulting in massive capitalisation and
  creation of tokens. During this period, some of the issues that
  blockchain technology faces were displayed. One of these issues is
  blockchain interoperability, the property of distinct blockchains to
  interact efficiently with each other. Despite its great importance,
  this field has not been addressed until recently. To date,
  cryptocurrencies are lacking a commonly accepted protocol that enables
  distributed interoperability. Such a protocol would be very useful to
  blockchain technology, since it would allow users to variously utilize
  features of different blockchains. For example, one can store their
  funds at Bitcoins, and convert them to Ether to make a payment,
  benefiting from lower transaction fees.

  A crosschain protocol would enable two main operations
  \begin{itemize}
  \item
    Crosschain trading: An entity that has deposits in a blockchain A,
    makes a payment to an entity at a blockchain B.
  \item
    Crosschain fund transfer: Entity transfers funds from blockchain A
    to blockchain B. After this operation, the funds no longer exist at
    blockchain A. The entity can decide to return any portion of the
    original amount to the blockchain of origin.
  \end{itemize}

  Currently, this operation is only available to the users via third
  party applications, such as multi-currency wallets. This treatment
  certainly opposes the nature of blockchain, which is a decentralized
  construction. This motivated us to create a solution that provides
  cheap, trust-less crosschain operations.

  \subsection{Objectives}

  In this paper, we focus on recent research in the area of crosschains.
  In particular, we make use of Non-Interactive Proofs of Proof of Work
  (NIPoPoWs)(ref), which enables the compression of a chain to its
  poly-logarothmic size. NIPoPoWs is the main building block of our
  solution in order to prove to blockchain A that an event happened in
  blockchain B.

  We were based on the work done by Giorgos Christoglou et al.(ref),
  which was the first ever implementation of crosschain events
  verification, and was used as a starting point for our solution. The
  work of Giorgos et al. focuses on verifying Bitcoin blockchain events
  in the Ethereum blockchain. In order to provide this functionality, a
  NIPoPoW verifier was developed in solidity(ref), the programming
  language of Ethereum blockchain. This solution, however, is impossible
  to be applied in a real blockchain due to extensive gas usage and
  severe security issues.

  A series of keen observations, the application of gas-efficient
  practices, and the utilization of modern solidity features, allowed us
  to repair previous security vulnerabilities and deliver a secure,
  superlight Bitcoin client in solidity, which can be applied to a real
  blockchain.

  Our target is to satisfy the following criteria:
  \begin{itemize}
  \item
    Applicability: The protocol should be applicable to the realistic
    blockchain.
  \item
    Security: The protocol should be secure against any adversarial
    attack.
  \item
    Trustless: The protocol should not have dependencies at any
    third-party
  \item
    Cheap: The protocol should be cheaper than the current state of the
    art technologies. This would make trust-less crosschain transactions
    more popular and affordable.
  \end{itemize}

  \pagebreak

  \section{Background}

  Relevant technologies

  \subsection{Primitives}

  Describe primitives

  \subsection{Bitcoin}

  Describe Bitcoin blockchain

  \subsection{Ethereum}

  Describe Ethereum blockchain

  \subsubsection{Solidity}

  Describe the use of solidity language

  \subsubsection{Smark contracts}

  Describe the use of smart contracts

  \subsubsection{Ethereum Virtual Machine}

  The Ethereum Virtual Machine (EVM) is a sandboxed virtual stack
  embedded within each full Ethereum node, responsible for executing
  contract bytecode. Contracts are typically written in higher level
  languages, like Solidity, then compiled to EVM bytecode.

  This means that the machine code is completely isolated from the
  network, filesystem or any processes of the host computer. Every node
  in the Ethereum network runs an EVM instance which allows them to
  agree on executing the same instructions. The EVM is Turing complete,
  which refers to a system capable of performing any logical step of a
  computational function. JavaScript, the programming language which
  powers the worldwide web, widely uses Turing completeness.

  Ethereum Virtual Machines have been successfully implemented in
  various programming languages including C++, Java, JavaScript, Python,
  Ruby, and many others.

  The EVM is essential to the Ethereum Protocol and is instrumental to
  the consensus engine of the Ethereum system. It allows anyone to
  execute code in a trustless ecosystem in which the outcome of an
  execution can be guaranteed and is fully deterministic (i.e.)
  executing smart contracts.

  \subsection{Non-Interactive Proofs Of Proof Of Work}

  Describe the rationale behind NIPoPoWs, what they provide

  \subsubsection{Prefix Proofs}

  Describe prefix proofs

  \subsubsection{Suffix Proofs}

  Describe suffix proofs

  \subsubsection{Infix Proofs}

  Describe infix proofs

  \subsection{Forks}

  Soft, hard and velvet fork

  \section{Implementation}

  In this section, the implementation steps are described. We also
  outline the difficulties we faced.

  \subsection{Setting up test environment}

  A testing environment is a setup of software to execute test cases.
  Test bed or test environment is configured as per the need of the
  application under test. Setting up a right test environment ensures
  software testing success.

  In the case of solidity applications, smart contracts are deployed on
  the EVM. The functionality of the deployed application is tested by
  making calls to the contract's functions and asserting the correctness
  of the results. In order to ensure that the contract always operates
  appropriately, all test-cases must be covered. A common practice for
  developers is to create a local, in-memory blockchain to deploy and
  test applications.

  There is variety of EVMs one can use. We experimented with are
  Ganache,Geth and Py-EVM.

  \subsubsection{Ganache}

  \subsubsection{Geth}

  \subsubsection{Py-EVM}

  \subsection{Targeting vulnerabilities and costly functionalities}

  \begin{itemize}
  \item
    0. Knowledge that extensive gas usage is due to large storage
    allocation
  \item
    1. Make old contract compatible with the last version of solidity
    compiler
  \item
    2. Create tests for all functionalities
  \item
    3. Identify and fix vulnerabilities

    Gas was very high due to storage variables

    Contract is vulnerable to premining
  \item
    4. Profile contract gas usage
  \item
    5. Identify expensive operations
  \end{itemize}

  \subsection{Fixing vulnerabilities and restricting gas usage}

  \begin{itemize}

  \item
    Add verification of genesis
  \item
    Reconsider DAG and ancestors methodology and redesign submission and
    contest schema
  \item
    Submit doesn't need to save in the proof in storage. Contest can
    provide the existing proof during contesting.

  \item
    Use subset: P1\{:lca\} $>$ P2\{:lca\}, P1 is the existing, P2 is the
    contesting

    This prevents expensive storage for DAG and ancestors.

  \item
    Even the check of subset can be skipped. The contesting proof cannot
    benefit from pre-lca malformed proof. If Pa is valid, then we just
    need to check Pb[lca:]

  \item
    This is not vulnerable to DOS attacks

  \item
    Observe the network and contest with the appropriate proof if
    submitted is not correct.

  \end{itemize}

  \section{Results}

  \section{Conclusion}

  \section{Future Work}

\end{document}
