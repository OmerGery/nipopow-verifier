\chapter{Introduction}

Bitcoin~\cite{nakamoto} is a form of decentralized money. Before Bitcoin was
invented, the only way to use money digitally was through an intermediary like
a bank. However, Bitcoin changed this by creating a decentralized form of
currency that individuals can trade directly without the need for an
intermediary. Each Bitcoin transaction is validated and confirmed by the entire
Bitcoin network.  There is no single point of failure so the system is
virtually impossible to shut down, manipulate or control.

The person (or group of people, as many think) behind Bitcoin, is known by the
name Shatoshi Nakamoto. Shatoshi put forth a construction that nowadays some
consider one of the most important achievements of our age, all fitting into a
9-page paper.  Bitcoin was published in November 2008, shortly followed by the
initiation of the Bitcoin network in January 2009 and is the first ever secure
and trust-less currency.

One of the by-products of the Bitcoin is blockchain. Blockchain technology was
created by fusing already existing technologies like cryptography, proof of
work and decentralized network architecture together in order to create a
system that can reach decisions without a central authority. There was no
“blockchain technology” before Bitcoin was invented, but once Bitcoin became a
reality, people started noticing how and why it works and named this
construction blockchain. Blockchain constitutes the very core of Bitcoin.
Later, it was realized that a currency like Bitcoin is just one of the
utilizations of the blockchain technology.

Ethereum~\cite{wood, buterin} was first proposed in late 2013 and then brought
to life in 2014.  Ethereum is a blockchain network that, apart from its digital
currency, Ether, hosts decentralized programs. These decentralized apps
(Dapps), or smart contracts, are written in Solidity~\cite{solidity}, the
programming language of Ethereum and yield to no single person control, not
even to their author.  The Ethereum platform is fully decentralized and
consists of thousands of independent computers running it. Once a program is
deployed to the Ethereum network it will be executed as written, hence the
famous phrase: "code is law".  Ethereum is a network of computers that together
combine into one powerful, decentralized supercomputer. Ethereum is often
characterized as the second era of blockchain networks.

\section{Motivation}

With the passing of time, new cryptocurrencies, altcoins as they are called in
the cryptocurrency folklore, are created every day. Some altcoins bring new
features to the cryprocurrency market and are accepted by the community, even
becoming popular. As of April 2020, there were over 5.392 cryptocurrencies with
a total market capitalisation of \$201 billion.

A newcomer to this world of distributed coins would possibly expect that there
must be some kind of established protocol for all these distinct blockchain to
interact; a way for Alice, who keeps her funds in Bitcoins, to transfer an
amount to Bob, who keeps his funds in Ether and vice-versa\footnote{The
    transfer of an amount from one chain to another is called one-way peg, and
the transfer of funds back to the original chain is called two-way peg.}.
In reality, the problem of blockchain interoperability had not been researched
until recently, and, to date, there is still no commonly accepted decentralized
protocol that enables interactions across blockchains, the so-called crosschain
operations.

\bigbreak
In general, crosschain-enabled blockchains A, B would satisfy the
following:

\begin{itemize}

    \item Crosschain trading: Alice with deposits in blockchain A, can make a
        payment to Bob at blockchain B.

    \item Crosschain fund transfer: Alice can transfers her funds from
        blockchain A to blockchain B. After this operation, the funds no longer
        exist in blockchain A. Alice can later decide to transfer any portion
        of the original amount to the blockchain of origin.

\end{itemize}

Currently, crosschain operations are only available to the users via
third-party applications, such as multi-currency wallets. It is obvious that
this centralized treatment opposes the nature of the blockchain and the
introspective of decentralized currencies. This contradiction motivated us to
create a solution that enables cheap and trust-less crosschain operations.

\section{Rationale}

In order to perform crosschain operations, mechanism that allows users of
blockchain A to discover events that have occurred in chain B, such as settled
transactions, must be introduced. One tricky aspect is to ensure the atomicity
of such operations, which require that either the transactions take place in
\emph{both} chains, or in \emph{neither}.  This is achievable through atomic
swaps~\cite{tiernolan,herlihy2018atomic}.  However, atomic swaps provide
limited functionality in that they do not allow the generic transfer of
information from one blockchain to a smart contract in another. For many
applications, a richer set of functionalities is
needed~\cite{pow-sidechains,derivatives}.  To communicate the fact that an
event took place in a source blockchain, a na\"ive approach is to have users
relay all the source blockchain blocks to a smart contract residing in the
target chain, which functions as a client for the remote chain and validates
all incoming information~\cite{btcrelay}.  This approach, however, is
impractical because a sizable amount of storage is needed to host entire chains
as they grow in time. As of June 2020, Bitcoin~\cite{nakamoto} chain spans
roughly 245 GB, and Ethereum~\cite{wood, buterin} has exceeded 300
GB\footnote{Size of blockchain derived from \url{https://www.statista.com},
\url{https://etherscan.io}}.

One early solution to compress the extensive size of blockchain and improve the
efficient of a client is addressed by
Nakamoto~\cite{nakamoto} with the Simplified Payment Verification (SPV)
protocol. In SPV, only the headers of blocks are stored, saving a considerable
amount of storage.  However, even with this protocol, the process of
downloading and validating all block headers still demands a considerable
amount of resources since they grow linearly in the size of the blockchain.
In Ethereum, for instance, headers sum up to approximately 4.8
GB\footnote{Calculated as the number of blocks (10,050,219) times the size of
header (508 bytes). Statistics by \url{https://etherscan.io/}.} of data.
These numbers quickly become impractical when it comes to consuming and storing the
data within a smart contract.


Another idea to make chain A interact with chain B, is to provide a
cryptographic proof to chain B that an event occurred in chain A. Secure
cryptographic proofs are mathematical constructions that are easy to verify and
impossible for an adversary to forge and are broadly used in cryptography and
blockchain in particular. In order to be more efficient than SPV, the size of
these proof needs to be small related to the size of the blockchain. This way,
we are be able to create proofs for events in chain A and send it to chain B
for validation. If chain B supports smart contracts, like Ethereum, the proof
can be verified automatically and transparently \emph{on-chain}. Notice that no
third-party is involved through the entire process.

\section{Related Work}

NIPoPoWs were introduced by Kiayias, Miller and Zindros~\cite{nipopows} and
their application to cross-chain communication was described in follow-up
work~\cite{pow-sidechains}, but only theoretically and without providing an
implementation. A few cryptocurrencies already include built-in NIPoPoWs
support, namely ERGO~\cite{ergo}, Nimiq~\cite{nimiq}, and
WebDollar~\cite{webdollar}; these chains can natively function as
\emph{sources} in cross-chain protocols.

Christoglou~\cite{gglou} provided a Solidity smart contract which is the first
implementation of crosschain events verification based on NIPoPoWs, where
proofs are submitted and checked for their validity, marking the first
implementation of an \emph{on-chain} verifier. This solution, however, is
impractical due to extensive usage of resources, widely exceeding the Ethereum
block gas limit.

Other attempts have been made to address the verification of
Bitcoin transactions to the Ethereum blockchain, most notably BTC
Relay~\cite{btcrelay}, which requires storing a full copy of all Bitcoin block
headers within the Ethereum chain.

\section{Our contributions}

Notably, no practical implementation for an on-chain superlight
clients exists to date. In this paper, we focus on constructing a practical
client for superblock NIPoPoWs.
For the implementation of our client, we refine the
NIPoPoW protocol based on a series of keen observations. These refinements
allow us to leverage useful techniques that construct a practical solution for
proof verification. We believe this achievement is a decisive and required step towards
establishing NIPoPoWs as the standard protocol for cross-chain communication.
A summary of our contributions in this paper is as follows:
\begin{enumerate}
\item We develop the first on-chain decentralized client that securely verifies
crosschain events and is practical. Our client establishes a trustless and
efficient solution to the interoperability problem. We implement\footnote{Our
implementation, unit tests and experiments can be found in
\url{https://github.com/sdaveas/nipopow-verifier}.} our client
in Solidity, and we verify Bitcoin events to the Ethereum blockchain. The
security assumptions we make are no other than
SPV's~\cite{eclipse, eclipse-ethereum}.
\item We present a novel pattern which we term \emph{hash-and-resubmit}. Our
pattern significantly improves performance of Ethereum smart
contracts~\cite{wood, buterin} in terms of gas consumption by utilizing the
\emph{calldata} space of Ethereum blockchain to eliminate high-cost storage
operations.
\item We create an \emph{optimistic} schema which we incorporate into the design
of our client. This design achieves significant performance improvement by
replacing linear complexity verification of proofs with constant complexity
verification.
\item We demonstrate that superblock NIPoPoWs are practical,
making it the first efficient cross-chain primitive.
\item We present a cryptoeconometric analysis of NIPoPoWs.
We provide concrete values for the collateral/liveness trade-off.
\end{enumerate}

Our implementation meets the following requirements:
\begin{enumerate}
\item \textbf{Security}: The client implements a provably secure protocol.
\item \textbf{Decentralization}: The client is not dependent on trusted third-parties
and operates in a transparent, decentralized manner.
\item \textbf{Efficiency}: The client complies with environmental constraints, i.e.\
block gas limit and calldata size limit of the Ethereum blockchain.
\end{enumerate}

We selected Bitcoin as the source blockchain as it the most popular cryptocurrency,
and enabling crosschain transactions in Bitcoin is beneficial to the
majority of the blockchain community. We selected Ethereum as the target blockchain
because, besides its popularity, it supports smart contracts, which is a
requirement in order to perform on-chain verification. We note here that
prior to Bitcoin events being consumable in Ethereum, Bitcoin requires a
velvet fork~\cite{velvet-fork}, a matter treated in a separate line of
work~\cite{velvet-nipopows}.

\bigbreak Some applications that demonstrate the usage of our client are:

\begin{itemize}
    \item{Application \#1}
    \item{Application \#2}
    \item{Application \#3}
\end{itemize}

\section{Structure}

\todo{Refine this}

In Section 2 we describe the blockchain technologies that are relevant to our
work. In Section 3 we put forth the \emph{hash-and-resubmit} pattern. We
demonstrate the improved performance of smart contracts using the pattern, and
how it is incorporated into our client. In Section 4, we present an alteration
to the NIPoPoW protocol that enables the elimination of look-up structures.
This allows for efficient interactions due to the considerably smaller size of
dispatched proofs. In Section 5, we put forth an optimistic schema that
significantly lowers the complexity of a proof's structural verification from
linear to constant, by introducing a new interaction which we term
\emph{dispute phase}. Furthermore, we present a technique that leverages the
dispatch of a constant number of blocks in the contest phase. Finally, in
Section 6, we present our cryptoeconomic analysis on our client and establish
the monetary value of collateral parameters.
