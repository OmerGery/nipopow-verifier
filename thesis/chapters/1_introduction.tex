\chapter{Introduction}

\section{Motivation}

Bitcoin is a form of decentralized money. Before Bitcoin was invented, the only
way to use money digitally was through an intermediary like a bank. However,
Bitcoin changed this by creating a decentralized form of currency that
individuals can trade directly without the need for an intermediary. Each
Bitcoin transaction is validated and confirmed by the entire Bitcoin network.
There is no single point of failure so the system is virtually impossible to
shut down, manipulate or control.

The person (or group of people, as many think) behind Bitcoin, is known by the
name Shatoshi Nakamoto. Shatoshi put forth a construction that nowadays some
consider one of the most important achievements of our age, all fitting into a
9-page paper.  Bitcoin was published in November 2008, shortly followed by the
initiation of the Bitcoin network in January 2009 and is the first ever secure
and trust-less currency.

One of the by-products of the Bitcoin is blockchain. Blockchain technology was
created by fusing already existing technologies like cryptography, proof of
work and decentralized network architecture together in order to create a
system that can reach decisions without a central authority. There was no
“blockchain technology” before Bitcoin was invented, but once Bitcoin became a
reality, people started noticing how and why it works and named this
construction blockchain. Blockchain constitutes the very core of Bitcoin.
Later, it was realized that a currency like Bitcoin is just one of the
utilizations of blockchain technology.

Ethereum was first proposed in late 2013 and then brought to life in 2014.
Ethereum is a blockchain network that, apart from its digital currency, Ether,
can host decentralized programs also known as Dapps - decentralized apps - also
called smart contracts. Smart contracts are written in Solidity, the
programming language of Ethereum and yield to no single person controls, not
even to their author.  The Ethereum platform is fully decentralized and
consists of thousands of independent computers running it. Once a program is
deployed to the Ethereum network it will be executed as written, hence the
famous phrase: "code is law". Ethereum is a network of computers that together
combine into one powerful, decentralized supercomputer. Ethereum is often
characterized as the second era of blockchain networks.

With the passing of time, new cryptocurrencies, altcoins as they are called in
the cryptocurrency folklore, are created every day. Some altcoins bring new
features to the cryprocurrency market and are accepted by the community, even
becoming popular. After Bitcoin and Ethereum, the most important blockchains in
terms of capitalization are XRP, Tether, Bitcoin Cash and Litecoin. As of April
22, there are over 5.392 cryptocurrencies with a total market capitalisation of
\$201 billion.

In this world of distributed coins, a newcomer to the cryptocurrency ecosystem
would possibly expect that there must be some kind of established way for all
these distinct blockchain to interact; a way for Alice, who keeps her funds in
Bitcoins to transfer an amount to Bob, who keeps his funds in Ether (one-way
peg) and vice-versa (two-way peg).

In reality, the problem of blockchain interoperability had not been researched
until recently, and, to date, there is still no commonly accepted decentralized
protocol that enables interactions between blockchains, the so-called
crosschain operations.  In general, crosschain-enabled blockchains A, B would
satisfy the following:

\begin{itemize}

    \item Crosschain trading: Alice with deposits in blockchain A, makes a
        payment to Bob at blockchain B.

    \item Crosschain fund transfer: Alice transfers her funds from blockchain A
        to blockchain B. After this operation, the funds no longer exist at
        blockchain A. Alice can later decide to transfer any portion of the
        original amount to the blockchain of origin.

\end{itemize}

Currently, crosschain operations are only available to the users via
third-party applications, such as multi-currency wallets. It is obvious that
this centralized treatment conflicts with the nature of the blockchain and the
introspective of decentralized currencies. This contradiction motivated us to
create a solution that enables cheap and trust-less crosschain operations.

\section{Rationale}

The essence of crosschain operations is to make known to chain A that an event
occurred to chain B, such that a payment tool place. One trivial way for Alice
to determine if an event took place in chain A and register it to chain B is to
participate at both chains. But this way is very inefficient because of the
blockchain size as it grows with time. At the moment, Bitcoin is 242.39 GB and
Ethereum has exceeded 1 TB. Naturally, it is impractical for every user to
accommodate this size of data.

An early solution to the extensive space of blockchain was given by Satoshi,
and is called Simplified Payment Verification (SPV) protocol. In SPV, only the
headers of blocks are stored, saving of a lot of storage. But even in this
case, Bitcoin headers have the accumulative size of 50 GB (80 bytes each).  A
mobile client needs several minutes, even hours to download all information
needed to function as an SPV client. Moreover, not all blockchains have
small-sized headers like Bitcoin. Block headers of Ethereum, for instance, are
508 bytes.

Another idea to make chain A interact with chain B, is to provide a
cryptographic proof to chain B that an event occurred in chain A. Secure
cryptographic proofs are mathematical constructions that are easy to verify and
impossible for an adversary to forge. In order to be efficient, the size of the
proof needs to be small in relation with the size of the blockchain. This way,
we are be able to create a proof for an event in chain A and send it to chain B
for validation. If chain B supports smart contracts, like Ethereum, then the
proof can be verified and no third-party is involved in the process.

\section{Previous Work}

Non-Interactive Proofs of Proof of Work (NIPoPoWs)(ref) are the fundamental
building block of our solution. This cryptographic primitive is provably secure
and provides succinctness. The size of the proofs is logarithmic to the size of
the underlying blockchain.

Giorgos Christoglou (ref), provided a Solidity smart contract which is the
first ever implementation of crosschain events verification based on NIPoPoWs.
This solution, however, is impossible to apply to the real blockchain due to
extensive usage of resources and important security issues.

\section{Our contributions}

\todo{Add later on: A series of keen observations, the application of
gas-efficient practices and the utilization of modern Solidity features led us
to the design of a new verifier architecture.}

We put forth the following contributions:

\begin{enumerate}[(a)]

    \item We developed the first ever client that verifies crosschain events
        and is applicable to the real blockchain. Our client establishes a
        secure, efficient, cheap and fully decentralized solution to the
        interoperability problem.

    \item We prove that NIPoPoWs can be applied to the real blockchain,
        \myworries{making the cryptographic primitive the first ever applied
        construction of succinct proofs to a real setting.}

\end{enumerate}

We selected the two most popular blockchains to demonstrate our solution:
Bitcoin as the source blockchain and Ethereum as the destination blockchain.

Our client meets the following requirements:
\begin{itemize}
    \item
        Security: The client is invulnerable against all adversarial attacks.
    \item
        Trust-less: The client is not dependent on third-party applications
        and operates in a fully decentralized manner.
    \item
        Applicability: The client can be utilized in the real blockchain and
        does not violate any of the environmental constraints.
    \item
        Cheap: The application is cheaper to use than the current state of the
        art technologies.
\end{itemize}

Some of the applications that demonstrate the usage of our client
are the following:

\begin{itemize}
    \item{Application \#1}
    \item{Application \#2}
    \item{Application \#3}
\end{itemize}

\section{Structure}

In Section 2 we background description of all relevant background
technologies and previous work. In Section 3 we put forth the implementation of
our client. In Section 4 we show our cryptoeconomic analysis and, finally, in
Section 5, we discuss applications of our client and future work.
