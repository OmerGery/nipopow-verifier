\begin{center}
    \section*{Abstract}
\end{center}

During the last years, significant effort has been put into enabling blockchain
interoperability, and several protocols have been proposed towards establishing
crosschain communication. Most notably, superlight clients, a new generation of
verifiers, have emerged. These clients demand only a poly-logarithmic number of
block headers in order to verify transactions, rather than the entire span of
the underlying chain. Albeit these constructions have been established
theoretically, no practical implementation exists to date. In this paper, we
focus on Non-Interactive Proofs of Proof of Work (NIPoPoWs), a probabilistic
structure based on superblocks that is provably secure and provides succinct
proofs of events in a blockchain. In particular, we discuss a gas-efficient
implementation for the verification of NIPoPoWs in Solidity, enabling
crosschain events from Bitcoin to Ethereum. We explore patterns and techniques
that considerably reduce gas consumption, and may have applications to other
smart contracts. We introduce a pattern that we term ``hash-and-resubmit'' that
eliminates persistent storage almost entirely, leading to significant increase
of performance. Furthermore, we alleviate the burden of expensive on-chain
operations, which we transfer off-chain, and we make use of an optimistic
schema that replaces functionalities of linear complexity with constant
operations. Lastly, we make a cryptoeconomic analysis, and set concrete values
regarding the cost that comes with using our client. Our implementation in
Solidity is accompanied by thorough unit tests. We display the performance gain
of our solution compared to previous work, and we mitigate security issues we
encountered such as premining.

\newpage

\begin{center}
    \section*{Περίληψη}
\end{center}

Τα τελευταία χρόνια, έχει καταβληθεί σημαντική προσπάθεια για να καταστεί
δυνατή η διαλειτουργικότητα των blockchain και αρκετά πρωτόκολλα έχουν προταθεί
για τη καθιέρωση της επικοινωνίας μεταξύ διαφορετικών αλυσίδων. Μια
αξιοσημείωτη τεχνολογία που έχει αυτό το στόχο είναι οι “υπερ-ελαφρείς”
πελάτες, μια νέα γενιά επαληθευτών. Αυτοί οι πελάτες απαιτούν μόνο έναν
πολυ-λογαριθμικό αριθμό κεφαλίδων μπλοκ για να επαληθεύσουν συναλλαγές, και όχι
τις κεφαλίδες ολόκληρης της υποκείμενης αλυσίδας. Αν και αυτές οι κατασκευές
έχουν γερά θεωρητικά θεμέλια, δεν έχει γίνει πρακτική εφαρμογή τους μέχρι
σήμερα. Σε αυτή την εργασία, εστιάζουμε στις Μη-Διαδραστικές Αποδείξεις
Απόδειξης Εργασίας (Non-Interactive Proofs of Proof of Work - NIPoPoWs), μια
πιθανοτική κατασκευή που βασίζεται στην λογική των superblock, είναι
αποδεδειγμένα ασφαλής και παρέχει αποδείξεις γεγονότων σε ένα Blockchain που
είναι σύντομες και περιεκτικές. Συγκεκριμένα, παρουσιάζουμε μια αποδοτική ως
προς τη κατανάλωση gas υλοποίηση σε Solidity για την επαλήθευση των NIPoPoW,
επιτρέποντας τη γνωστοποίηση γεγονότων από την αλυσίδα του Bitcoin στη αλυσίδα
του Ethereum. Εισάγουμε μοτίβα και τεχνικές που μειώνουν σημαντικά την
κατανάλωση του gas και ενδέχεται να έχουν εφαρμογές σε άλλα έξυπνα συμβόλαια.
Παρουσιάζουμε ένα μοτίβο που ονομάζουμε ``hash-and-resubmit'', το οποίο εξαλείφει
σχεδόν εξ ολοκλήρου τη μόνιμη μνήμη του έξυπνου συμβολαίου μας, οδηγώντας σε
σημαντική αύξηση της απόδοσης. Επιπλέον, μετριάζουμε το βάρος δαπανηρών πράξεων
στην αλυσίδα, τις οποίες μεταφέρουμε εκτός αλυσίδας, και χρησιμοποιούμε ένα
αισιόδοξο σχήμα (optimistic scheme) που αντικαθιστά λειτουργίες γραμμικής
πολυπλοκότητας με λειτουργίες σταθερού χρόνου. Τέλος, κάνουμε μια
κρυπτο-οικονομική ανάλυση και καθορίζουμε συγκεκριμένες τιμές σχετικά με το
κόστος της χρήσης του πελάτη μας. Η εφαρμογή μας στη Solidity συνοδεύεται από
ενδελεχή unit tests. Παρουσιάζουμε επίσης τη διαφορά απόδοσης της λύσης μας σε
σύγκριση με προηγούμενη σχετική εργασία και εξαλείφουμε τα ζητήματα ασφάλειας
που συναντήσαμε, όπως το premining.
