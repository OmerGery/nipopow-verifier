\chapter{Introduction}

\section{Motivation}

Bitcoin is a form of decentralized money. Before Bitcoin was invented, the only
way to use money digitally was through an intermediary like a bank. However,
Bitcoin changed this by creating a decentralized form of currency that
individuals can trade directly without the need for an intermediary. Each
Bitcoin transaction is validated and confirmed by the entire Bitcoin network.
There is no single point of failure so the system is virtually impossible to
shut down, manipulate or control.

The person (or group of people, as many think) behind Bitcoin, is known by the
name Shatoshi Nakamoto. Shatoshi put forth a construction that nowadays some
consider one of the most important achievements of our age, all fitting into a
9-page paper.  Bitcoin was published in November 2008, shortly followed by the
initiation of the Bitcoin network in January 2009 and is the first ever secure
and trust-less currency.

One of the by-products of the Bitcoin is blockchain. Blockchain technology was
created by fusing already existing technologies like cryptography, proof of
work and decentralized network architecture together in order to create a
system that can reach decisions without a central authority. There was no
“blockchain technology” before Bitcoin was invented, but once Bitcoin became a
reality, people started noticing how and why it works and named this
construction blockchain. Blockchain constitutes the very core of Bitcoin.
Later, it was realized that a currency like Bitcoin is just one of the
utilizations of the blockchain technology.

Ethereum was first proposed in late 2013 and then brought to life in 2014.
Ethereum is a blockchain network that, apart from its digital currency, Ether,
hosts decentralized programs. These decentralized apps (Dapps), or smart
contracts, are written in Solidity, the programming language of Ethereum and
yield to no single person control, not even to their author.  The Ethereum
platform is fully decentralized and consists of thousands of independent
computers running it. Once a program is deployed to the Ethereum network it
will be executed as written, hence the famous phrase: "code is law". Ethereum
is a network of computers that together combine into one powerful,
decentralized supercomputer. Ethereum is often characterized as the second era
of blockchain networks.

With the passing of time, new cryptocurrencies, altcoins as they are called in
the cryptocurrency folklore, are created every day. Some altcoins bring new
features to the cryprocurrency market and are accepted by the community, even
becoming popular. After Bitcoin and Ethereum, the most important blockchains in
terms of capitalization are Ripple, Tether, Bitcoin Cash and Litecoin. As of
April 2020, there were over 5.392 cryptocurrencies with a total market
capitalisation of \$201 billion.

A newcomer to this world of distributed coins would possibly expect that there
must be some kind of established protocol for all these distinct blockchain to
interact; a way for Alice, who keeps her funds in Bitcoins, to transfer an
amount to Bob, who keeps his funds in Ether and vice-versa\footnote{The
    transfer of an amount from one chain to another is called one-way peg, and
the transfer of funds back to the original chain is called two-way peg.}.
In reality, the problem of blockchain interoperability had not been researched
until recently, and, to date, there is still no commonly accepted decentralized
protocol that enables interactions across blockchains, the so-called crosschain
operations.

\bigbreak
In general, crosschain-enabled blockchains A, B would satisfy the
following:

\begin{itemize}

    \item Crosschain trading: Alice with deposits in blockchain A, can make a
        payment to Bob at blockchain B.

    \item Crosschain fund transfer: Alice can transfers her funds from
        blockchain A to blockchain B. After this operation, the funds no longer
        exist in blockchain A. Alice can later decide to transfer any portion
        of the original amount to the blockchain of origin.

\end{itemize}

Currently, crosschain operations are only available to the users via
third-party applications, such as multi-currency wallets. It is obvious that
this centralized treatment opposes the nature of the blockchain and the
introspective of decentralized currencies. This contradiction motivated us to
create a solution that enables cheap and trust-less crosschain operations.

\section{Rationale}

To perform crosschain operations between two chains, there must be some
mechanism to communicate to chain A that an event occurred to chain B, such
that a payment tool place. One trivial way for Alice to determine if an event
took place in chain A and register it to chain B is to participate in both
chains.  But this way is very inefficient because the storage needed to store
the blockchain is sizable, and it grows with time. At the moment, Bitcoin is
242.39 GB and Ethereum has exceeded 1 TB. Naturally, it is impractical for
every user to accommodate this size of data.

An early solution to the extensive space of blockchain was given by Satoshi,
and is called Simplified Payment Verification (SPV) protocol. In SPV, only the
headers of blocks are stored, saving of a lot of storage. However, even with
this protocol, Bitcoin headers have the total size of 50 GB (80 bytes each). A
mobile client needs several minutes, even hours to download all information
needed to function as an SPV client. Moreover, not all blockchains have
small-sized headers like Bitcoin. Block headers of Ethereum, for instance, are
508 Bytes and for different altcoins they span several MB of data. SPV clients
lead to unpleasant user experience at some cases, while they are completely
impractical in others.

Another idea to make chain A interact with chain B, is to provide a
cryptographic proof to chain B that an event occurred in chain A. Secure
cryptographic proofs are mathematical constructions that are easy to verify and
impossible for an adversary to forge and are broadly used in cryptography and
blockchain in particular. In order to be more efficient than SPV, the size of
these proof needs to be small related to the size of the blockchain. This way,
we are be able to create proofs for events in chain A and send it to chain B
for validation. If chain B supports smart contracts, like Ethereum, the proof
can be verified automatically and transparently \emph{on-chain}. Notice that no
third-party is involved through the entire process.

\section{Related Work}

Non-Interactive Proofs of Proof of Work (NIPoPoWs)(ref) is the fundamental
building block of our solution. This cryptographic primitive is provably secure
and provide succinct proofs regarding the occurrence of an event in a chain.
The size of the proofs is logarithmic to the size of the underlying blockchain,
which means that they are growing slowly compared to the blockchain growth
rate.

Christoglou (ref), has provided a Solidity smart contract which is the first
ever implementation of crosschain events verification based on NIPoPoWs, where
proofs are submitted and checked their validity. This solution, however, is
impossible to apply to the real blockchain due to extensive usage of resources
and important security vulnerabilities.

A different aspect to solve interoperability is a protocol introduced by Summa
team. In this work users waits for a transaction to be buried under several
blocks which implies that the transaction must belong to the real chain, and
thus be valid. This treatment enables fast and cheap crosschain capabilities.
But this solution has not been proved cryptographically secure. In fact, an
attack has been laid out that makes the protocol vulnerable to non-rational
adversaries.

\section{Our contributions}

% \todo{Add later on: A series of keen observations, the application of
% gas-efficient practices and the utilization of modern Solidity features led us
% to the design of a new verifier architecture.}

We put forth the following contributions:

\begin{enumerate}[(a)]

    \item We developed the first ever decentralized client that securely
        verifies crosschain events and is applicable to the real blockchain.
        Our client establishes a safe, cheap and trust-less solution to the
        interoperability problem. Our client is implemented in Solidity and
        verifies Bitcoin events to the Ethereum blockchain.

    \item We prove via application that NIPoPoWs can be utilized in the real
        blockchain, making the cryptographic primitive the first
        ever applied construction of succinct proofs to a real setting.

\end{enumerate}

\bigbreak
Our implementation meets the following requirements:
\begin{enumerate}[(a)]

    \item
        Security: The client is invulnerable against all adversarial attacks.

    \item Is trust-less: The client is not dependent on third-party
        applications and operates in a fully transparent, decentralized manner.

    \item Applicability: The client can be utilized in the real blockchain and
        comply with all environmental constraints, i.e.\ block gas limit and
        calldata size of Ethereum blockchain.

    \item Is cheap: The application is cheaper to use than the current state of
        the art technologies.

\end{enumerate}

We selected Bitcoin as source blockchain as it the most used cryptocurrency and
enabling crosschain transactions in Bitcoin is beneficial to the vast majority
of blockchain community. We selected Ethereum as the target blockchain because
it is also very popular to the community and it supports smart contracts which
is a requirement in order to perform on-chain verification.

\bigbreak Some applications that demonstrate the usage of our client are:

\begin{itemize}
    \item{Application \#1}
    \item{Application \#2}
    \item{Application \#3}
\end{itemize}

\section{Structure}

In Section 2 we description of all relevant background technologies and
previous work. In Section 3 we put forth the implementation of our client. In
Section 4 we show our cryptoeconomic analysis and, finally, in Section 5, we
discuss applications of our client and future work.
