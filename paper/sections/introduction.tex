\section{Introduction}

Blockchain interoperability\ifanonymous\else~\cite{dionyziz}\fi is the ability of distinct
blockchains to communicate.  This \emph{crosschain}~\cite{crosschain-sok,gtklocker} communication enables useful
features across blockchains such as the transfer of assets from one chain to
another (one-way peg)~\cite{burn} and back (two-way peg)~\cite{pos-sidechains},
as well as the generic passing of information from chain to chain~\cite{pow-sidechains}. To date,
there is no commonly accepted decentralized protocol that enables cross-chain
transactions.

\noindent

In general, crosschain-enabled blockchains A, B support the following
operations:

\begin{itemize}
\item Crosschain trading: a user with deposits in blockchain A, makes a
    payment to a user in blockchain B.
\item Crosschain fund transfer: a user transfers her funds from blockchain
    A to blockchain B. After the transfer, these funds no longer exist in
    blockchain A. The user can later decide to transfer any portion of the
    original amount to the blockchain of origin.
\end{itemize}

In order to perform crosschain operations, mechanism that
allows users of blockchain A to discover events that have occurred in chain B,
such as settled transactions, must be introduced. One tricky aspect is to ensure the atomicity of such
operations, which require that either the transactions take place in \emph{both}
chains, or in \emph{neither}.
This is achievable through atomic swaps~\cite{tiernolan,herlihy2018atomic}.
However, atomic swaps provide limited functionality in that they do not allow
the generic transfer of information from one blockchain to a smart contract in
another. For many applications, a richer set of
functionalities is needed~\cite{pow-sidechains,derivatives}.
To communicate the fact that an event took place in a source blockchain,
a na\"ive approach is to have users relay all the source blockchain blocks to a
smart contract residing in the target chain, which functions as a
client for the remote chain and validates all incoming information~\cite{btcrelay}.
This approach, however, is
impractical because a sizable amount of storage is needed to host entire
chains as they grow in time. As of June 2020, Bitcoin~\cite{nakamoto} chain
spans roughly 245 GB, and Ethereum~\cite{wood, buterin} has exceeded 300
GB\footnote{Size of blockchain derived from https://www.statista.com,
https://etherscan.io}.

One early solution to compress the extensive size of blockchain and improve the
efficient of a client is addressed by
Nakamoto~\cite{nakamoto} with the Simplified Payment Verification (SPV)
protocol. In SPV, only the headers of blocks are stored, saving a considerable
amount of storage.  However, even with this protocol, the process of
downloading and validating all block headers still demands a considerable
amount of resources since they grow linearly in the size of the blockchain.
In Ethereum, for instance, headers sum up to approximately 4.8
GB\footnote{Calculated as the number of blocks (10,050,219) times the size of
header (508 bytes). Statistics by https://etherscan.io/.} of data.
These numbers quickly become impractical when it comes to consuming and storing the
data within a smart contract.

Towards the goal of delivering more practical solutions for blockchain
transaction verification, a new generation of \emph{superlight}
clients has emerged~\cite{popow,nipopows,compactsuperblocks,flyclient}. In these
protocols, cryptographic proofs are generated, that prove the occurrence of
events in a blockchain. These proofs require only a polylogarithmic size of
data compared to the SPV model, resulting in better performance. By utilizing
superlight client protocols, a compressed proof for an event in chain A is
constructed and dispatched to chain B. If chain B supports smart contracts, the
proof is then verified automatically and transparently \emph{on-chain}. This
communication is realized without the intervention of trusted third-parties.
An interesting application of such a protocol is the communication between
Bitcoin and Ethereum and in particular the passing of Bitcoin events to Ethereum
smart contracts.

The first protocol in this family is the \emph{superblocks} Non-Interactive
Proofs of Proof-of-Work (NIPoPoWs) protocol. This cryptographic primitive is \emph{provably secure} and
provides \emph{succinct proofs} about the existence of an arbitrary event
in a chain. We leverage NIPoPoWs as the fundamental building block of
our solution.

\noindent
\textbf{Related Work.} NIPoPoWs
were introduced by Kiayias, Miller and Zindros~\cite{nipopows} and their
application to cross-chain communication was described in follow-up
work~\cite{pow-sidechains}, but only theoretically and without providing
an implementation. A few cryptocurrencies already
include built-in NIPoPoWs support, namely ERGO~\cite{ergo}, Nimiq~\cite{nimiq}, and
WebDollar~\cite{webdollar}; these chains can natively function as \emph{sources}
in cross-chain protocols.
Christoglou~\cite{gglou} provided a Solidity smart contract which is the first
implementation of crosschain events verification based on NIPoPoWs, where
proofs are submitted and checked for their validity, marking the first
implementation of an \emph{on-chain} verifier. This solution, however, is
impractical due to extensive usage of resources, widely exceeding the Ethereum
block gas limit.
Other attempts have been made to address the verification of Bitcoin
transactions to the Ethereum blockchain, most notably BTC
Relay~\cite{btcrelay}, which requires storing a full copy of all Bitcoin
block headers within the Ethereum chain.

\noindent
\textbf{Our contributions.}
Notably, no practical implementation for an on-chain superlight
clients exists to date. In this paper, we focus on constructing a practical
client for superblock NIPoPoWs.
For the implementation of our client, we refine the
NIPoPoW protocol based on a series of keen observations. These refinements
allow us to leverage useful techniques that construct a practical solution for
proof verification. We believe this achievement is a decisive and required step towards
establishing NIPoPoWs as the standard protocol for cross-chain communication.
A summary of our contributions in this paper is as follows:
\begin{enumerate}
\item We develop the first on-chain decentralized client that securely verifies
crosschain events and is practical. Our client establishes a trustless and
efficient solution to the interoperability problem. We implement\footnote{Our
implementation, unit tests and experiments can be found in https://github.com/superlightBitcoinClientInSolidity/verifier. The implementation will be released as open source software and deanonymized
in the proceedings version of this paper.} our client
in Solidity, and we verify Bitcoin events to the Ethereum blockchain. The
security assumptions we make are no other than
SPV's~\cite{eclipse, eclipse-ethereum}.
\item We present a novel pattern which we term \emph{hash-and-resubmit}. Our
pattern significantly improves performance of Ethereum smart
contracts~\cite{wood, buterin} in terms of gas consumption by utilizing the
\emph{calldata} space of Ethereum blockchain to eliminate high-cost storage
operations.
\item We create an \emph{optimistic} schema which we incorporate into the design
of our client. This design achieves significant performance improvement by
replacing linear complexity verification of proofs with constant complexity
verification.
\item We demonstrate that superblock NIPoPoWs are practical,
making it the first efficient cross-chain primitive.
\item We present a cryptoeconometric analysis of NIPoPoWs.
We provide concrete values for the collateral/liveness trade-off.
\end{enumerate}

Our implementation meets the following requirements:
\begin{enumerate}
\item \textbf{Security}: The client implements a provably secure protocol.
\item \textbf{Decentralization}: The client is not dependent on trusted third-parties
and operates in a transparent, decentralized manner.
\item \textbf{Efficiency}: The client complies with environmental constraints, i.e.\
block gas limit and calldata size limit of the Ethereum blockchain.
\end{enumerate}

We selected Bitcoin as the source blockchain as it the most popular cryptocurrency,
and enabling crosschain transactions in Bitcoin is beneficial to the
majority of the blockchain community. We selected Ethereum as the target blockchain
because, besides its popularity, it supports smart contracts, which is a
requirement in order to perform on-chain verification. We note here that
prior to Bitcoin events being consumable in Ethereum, Bitcoin requires a
velvet fork~\cite{velvet-fork}, a matter treated in a separate line of
work~\cite{velvet-nipopows}.

% Bitcoin does not include interlink structures in blocks. The setting we refer
% to is a hard fork~\cite{hard-fork}, but we believe that a
% velvet~\cite{velvet-fork} fork can also be supported as it does not add
% significant overhead to the verifier as shown in ~\cite{andri}.

\noindent
\textbf{Structure.} In Section 2 we describe the blockchain technologies that
are relevant to our work. In Section 3 we put forth the
\emph{hash-and-resubmit} pattern. We demonstrate the improved performance of
smart contracts using the pattern, and how it is incorporated into our
client. In Section 4, we present an alteration to the NIPoPoW protocol that
enables the elimination of look-up structures. This allows for efficient
interactions due to the considerably smaller size of dispatched proofs. In
Section 5, we put forth an optimistic schema that significantly lowers the
complexity of a proof's structural verification from linear to constant, by
introducing a new interaction which we term \emph{dispute phase}. Furthermore,
we present a technique that leverages the dispatch of a constant number of blocks
in the contest phase. Finally, in Section 6, we present our cryptoeconomic analysis
on our client and establish the monetary value of collateral parameters.
