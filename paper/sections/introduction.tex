\section{Introduction}

Blockchain interoperability~\cite{dionyziz} is the ability of distinct
blockchains to communicate.  This \emph{cross-chain}~\cite{pow-sidechains,
pos-sidechains,burn,crosschain-sok, gtklocker} communication can enable useful
features across blockchains such as the transfer of asset from one chain to
another (one-way peg) and back (one-way peg)~\cite{pow-sidechains}. To date,
there is no commonly accepted decentralized protocol that enables cross-chain
transactions.  Currently, crosschain operations are only available to the users
via third-party applications, such as multi-currency wallets. However, this
treatment is opposing to the nature of decentralized currencies.

\noindent

In general, crosschain-enabled blockchains A, B would support the following
operations:

\begin{itemize}
\item Crosschain trading: A user with deposits in blockchain A, can make a
    payment to a user with funds in blockchain B.
\item Crosschain fund transfer: A user can transfers her funds from blockchain
    A to blockchain B. After this operation, the funds no longer exist in
    blockchain A. The user can later decide to transfer any portion of the
    original amount to the blockchain of origin.
\end{itemize}


\noindent

In order to perform crosschain operations, there must be some mechanism to
allow users of blockchain A discover events that occur in chain B, such that a
transaction occurred. A trivial way is to participate as a full node in both
chains. This approach, however, is impractical because a sizeable amount of
storage is needed to host entire chains as they grow with time. As of May 2020,
Bitcoin~\cite{nakamoto} chain spans roughly 245 GB and Ethereum~\cite{wood,
buterin} has exceeded 1 TB. Naturally, not all users are able to accommodate
this size of data especially if portable media are used, such as mobile phones.

An early solution to compress the extensive space of blockchain was given by
Nakamoto with the Simplified Payment Verification (SPV) protocol
~\cite{nakamoto}. In SPV, only the headers of blocks are stored, saving a
considerable amount of storage.  However, even with this protocol, the process
of downloading and validating all block headers can lead to unpleasant user
experience. In Ethereum, for instance, headers sum up to roughly 4 GB of data.
A mobile client needs several minutes, even hours to fetch all information
needed in order to function as an SPV client.

In order to deliver more practical solutions for blockchain verification, a new
generation of \emph{superlight}
clients~\cite{popow,nipopows,compactsuperblocks, flyclient} emerged. In these
protocols, cryptographic proofs are generated, which prove the occurrence of
events inside a blockchain. Better performance is achieved due to considerably
smaller size of proofs compared to the data needed in SPV. By utilizing
superlight client protocols, a compressed proof for an event in chain A is
constructed, and, if chain B supports smart contracts, the proof can then be
verified automatically and transparently \emph{on-chain}.  Note that, this
communication is realized without the intervention of third-party applications.
An interesting application of such a protocol is the communication between
Bitcoin and Ethereum.

\noindent

\textbf{Related Work.} We use Non-Interactive Proofs of Proof of Work
(NIPoPoWs)~\cite{nipopows} as the fundamental building block of our solution.
This cryptographic primitive is \emph{provably secure} and provides
\emph{succinct proofs} regarding the existence of an arbitrary event in a
chain. Contrary to the linear growth rate of the underlying blockchain,
NIPoPoWs span polylogarithmic size of blocks.

Christoglou~\cite{gglou}, has provided a Solidity smart contract which is the first
ever implementation of crosschain events verification based on NIPoPoWs, where
proofs are submitted and checked for their validity. This solution, however, is
impossible to apply to the real blockchain due to extensive usage of resources
and important security vulnerabilities such as premining.

\noindent

\textbf{Our contributions.} We put forth the following contributions:
\begin{enumerate}
\item We developed the first ever decentralized client that securely verifies
crosschain events and is applicable to the real blockchain. Our client
establishes a safe, cheap and trust-less solution to the interoperability
problem. We implemented our client in Solidity, and we verify Bitcoin events
to the Ethereum blockchain.
\item We present a novel pattern which we term \emph{hash-and-resubmit}. This
pattern improves the performance of Ethereum smart contracts~\cite{wood,
buterin} in terms of gas consumption by utilizing \emph{calldata} space to
eliminate high-cost storage operations.
\item We prove via application that NIPoPoWs can be utilized in the real
blockchain, making the cryptographic primitive the first ever provably secure
construction of succinct proofs which is applied in a real setting.
\item Optimistic vs non-optimistic constructions - Ask gtklocker
\end{enumerate}

\bigbreak
Our implementation meets the following requirements:
\begin{enumerate}
\item Security: The client is invulnerable against all adversarial attacks.
\item Is trust-less: The client is not dependent on third-party applications
and operates in a transparent, decentralized manner.
\item Applicability: The client can be utilized in the real blockchain and
comply with all environmental constraints, i.e.\ block gas limit and calldata
size of Ethereum blockchain.
\item Is cheap: The application is cheaper to use than the current state of the
art technologies.
\end{enumerate}

We selected Bitcoin as source blockchain as it the most used cryptocurrency and
enabling crosschain transactions in Bitcoin is beneficial to the vast majority
of blockchain community. We selected Ethereum as the target blockchain because
it is also very popular and it supports smart contracts, which is a requirement
in order to perform on-chain verification.

\noindent

\textbf{Structure.} In Section 2 describe all blockchain technologies which are
relevant to our work. In Section 3 we put forth .... In Section 4 we show
... and, finally, in Section 5, we discuss ...


