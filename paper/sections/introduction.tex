\section{Introduction}

\emph{Interoperability} is the ability of distinct blockchains to communicate.
These \emph{cross-chain} interactions can enable useful features across
blockchains such as the transfer of asset from one chain to another (one-way
peg) and back (one-way peg). To date, there is no commonly accepted
decentralized protocol that enables cross-chain transactions. Currently,
crosschain operations are only available to the users via third-party
applications, such as multi-currency wallets. However, this treatment is
opposing to the nature of the decentralized currencies.

\noindent

In general, crosschain-enabled blockchains A, B would support the following
operations:

\begin{itemize}
\item Crosschain trading: Alice with deposits in blockchain A, can make a
    payment to Bob at blockchain B.
\item Crosschain fund transfer: Alice can transfers her funds from blockchain A
    to blockchain B. After this operation, the funds no longer exist in
    blockchain A. Alice can later decide to transfer any portion of the
    original amount to the blockchain of origin.
\end{itemize}


\noindent

To perform crosschain operations, there must be some mechanism to discover from
chain A that an event occurred to chain B, such that a transaction occurred. A
trivial way is to participate as a node in both chains. This approach, however,
is very impractical because a sizeable amount of storage is needed to host
for the entire chain, as they grow with time. At the moment, Bitcoin chain
spans roughly 245 GB and Ethereum has exceeded the 1 TB. Naturally, not all
users have the convenience to accommodate this size of data.

An early solution to compress the extensive space of blockchain was given by
Nakamoto with the Simplified Payment Verification (SPV) protocol. In SPV, only
the headers of blocks are stored, saving a considerable amount of storage.
However, even with this protocol, Bitcoin headers sum up to 50 GB of data (80
bytes per header). A mobile client needs several minutes, even hours to fetch
all information needed in order to function as an SPV client. Moreover, not all
blockchains have small-sized headers like Bitcoin. Block headers of Ethereum,
for instance, are 508 Bytes and in other altcoins the size of each header is
several MB. Even with the use of SPV protocol, the process of downloading and
validating all block headers can lead to unpleasant user experience.

In order to provide a more practical solution than SPV clients, a new
generation of \emph{superlight clients} emerged. In these protocols,
cryptographic proofs are generated, which prove the occurrence of events inside
a blockchain. Better performance is achieved due to the considerably smaller
size of the proofs compared to the SPV protocol. By utilizing superlight client
protocols, a compressed proof for an event in chain A can be constructed, and,
if chain B supports smart contracts, the proof can be then verified
automatically and transparently \emph{on-chain}. Note that, this communication
is realized without the intervention of third-party applications. An
interesting application of such a protocol is the communication of Bitcoin
events to Ethereum.

\noindent

\textbf{Related Work.} Non-Interactive Proofs of Proof of Work (NIPoPoWs)(ref)
is the fundamental building block of our solution. This cryptographic primitive
is \emph{provably secure} and provides \emph{succinct proofs} regarding the
existence of an arbitrary event in a chain. Contrary to the linear growth rate
of the underlying blockchain, NIPoPoWs span logarithmic size of blocks

Christoglou (ref), has provided a Solidity smart contract which is the first
ever implementation of crosschain events verification based on NIPoPoWs, where
proofs are submitted and checked for their validity. This solution, however, is
impossible to apply to the real blockchain due to extensive usage of resources
and important security vulnerabilities.

A different methodology to solve interoperability is a protocol introduced by
Summa team. In this work, users waits for a transaction to be buried under
several blocks which implies that the transaction must belong to the real
chain, and thus be valid. This treatment enables fast and cheap crosschain
capabilities. But this solution has not been proved to be cryptographically
secure. In fact, an attack has been laid out that makes the protocol vulnerable
to non-rational adversaries.

\noindent

\textbf{Our contributions.} We put forth the following contributions:
\begin{enumerate}
\item We developed the first ever decentralized client that securely verifies
crosschain events and is applicable to the real blockchain. Our client
establishes a safe, cheap and trust-less solution to the interoperability
problem. Our client is implemented in Solidity and verifies Bitcoin events
on the Ethereum blockchain.
\item We prove via application that NIPoPoWs can be utilized in the real
blockchain, making the cryptographic primitive the first ever applied
construction of secure, succinct proofs in a real setting.
\item We present a novel pattern which we term \emph{hash-and-resubmit}. This
pattern improves the performance of smart contracts in terms of gas consumption
by utilizing \emph{calldata} space to eliminate high-cost storage operations.
\item Optimistic vs non-optimistic constructions ??
\end{enumerate}

\bigbreak
Our implementation meets the following requirements:
\begin{enumerate}
\item Security: The client is invulnerable against all adversarial attacks.
\item Is trust-less: The client is not dependent on third-party applications
and operates in a transparent, decentralized manner.
\item Applicability: The client can be utilized in the real blockchain and
comply with all environmental constraints, i.e.\ block gas limit and calldata
size of Ethereum blockchain.
\item Is cheap: The application is cheaper to use than the current state of the
art technologies.
\end{enumerate}

We selected Bitcoin as source blockchain as it the most used cryptocurrency and
enabling crosschain transactions in Bitcoin is beneficial to the vast majority
of blockchain community. We selected Ethereum as the target blockchain because
it is also very popular and it supports smart contracts, which is a requirement
in order to perform on-chain verification.

\noindent

\textbf{Structure.} In Section 2 describe all blockchain technologies which are
relevant to our work. In Section 3 we put forth .... In Section 4 we show
... and, finally, in Section 5, we discuss ...


