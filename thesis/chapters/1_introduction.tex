\chapter{Introduction}

\section{Motivation}

Digital coins are peer-to-peer currencies based on applied cryptography for the
validation of transactions. Most of them are based on blockchain(ref)
technology, a form of a decentralized database. In this database, a public
ledger is deployed which is stored and updated by thousands of users in absence
of supervision from public authorities.

In 2008, Bitcoin(ref), the first ever successful decentralized digital coin,
was invented by an unknown person or group of people using the name Satoshi
Nakamoto. A year after, the bitcoin network started, quickly followed by
several other digital coins, known as altcoins in the cryptocurrency folklore.
Usually, altcoins provide new, innovative features missing from the
cryptocurrency market. Popular altcoins are Ethereum(ref), which is the first
to provide smart contracts, Ripple(ref) which provides real-time payment
settlements and Litecoin(ref) which enables near-zero cost payments.

Over the last decade, cryptocurrencies gained attention from the public as an
increased number of users accept and trust decentralized transactions.
Specifically, in 2017, the popularity of cryptocurrencies rapidly grew,
resulting in massive capitalisation and creation of tokens. During this period,
some of the issues that blockchain ecosystem faces were displayed. One of these
issues is blockchain interoperability, the property of different blockchains to
efficiently interact with each other. Despite its great importance, this field
has not been addressed until recently. To date, cryptocurrencies are lacking a
commonly accepted protocol to communicate. Such a protocol would be very useful
to blockchain ecosystems, since it would allow users to utilize various
features of different blockchains. For example, to store their funds in
Bitcoins and convert them to Ether to make a payment, benefiting from lower
transaction fees and quicker transaction times.

The application of a crosschain protocol brings two main capabilities to the
users:
\begin{itemize}
    \item
        Crosschain trading: A user with deposits in blockchain A, makes a
        payment to a user at blockchain B.
    \item
        Crosschain fund transfer: A user transfers owning funds from
        blockchain A to blockchain B. After this operation, the funds no
        longer exist at blockchain A. The user can transfer any portion of the
        original amount to the blockchain of origin.
\end{itemize}

Currently, crosschain operations are only available to the users via third
party applications, such as multi-currency wallets. This centralized treatment
is contradicting to the nature of blockchain, and abolishes the decentralized
aspect of cryptocurrencies. This motivated us to create a solution that enables
cheap and trust-less crosschain operations.

\section{Previous Work}

Non-Interactive Proofs of Proof of Work (NIPoPoWs)(ref) are the fundamental
building block of our solution. This cryptographic primitive enables the
compression of blockchains and makes the occurrence of an event of blockchain A
known to blockchain B provable and efficient.

We refer to previous work by Giorgos Christoglou et al.(ref), which is the
first ever implementation of crosschain events verification based on NIPoPoWs.
This solution, however, is impossible to apply to the real blockchain due to
extensive usage of resources and important security issues.

\section{Our contributions}

\todo{Add later on: A series of keen observations, the application of
gas-efficient practices and the utilization of modern Solidity features led us
to the design of a new verifier architecture.}

We developed the first ever client that can perform crosschain operations and
is applicable to the real blockchain. Our client establishes a secure,
efficient, cheap and fully decentralized solution to the interoperability
problem.

We also provide plain evidence that NIPoPoWs can be applied to the real world,
\myworries{making the cryptographic primitive the first ever applied
construction of succinct proofs to a real setting.}

Our crosschain client meets the following requirements:
\begin{itemize}
    \item
        Security: The client is invulnerable against all adversarial attacks.
    \item
        Trust-less: The client is not dependent on third-party applications
        and operates in a fully decentralized manner.
    \item
        Applicability: The client can be utilized in the real blockchain and
        does not violate any of the environmental constraints.
    \item
        Cheap: The application is cheaper to use than the current state of the
        art technologies.
\end{itemize}

We selected the two most popular blockchains to demonstrate our solution: (a)
Bitcoin as the source blockchain and (b) Ethereum as the destination
blockchain. Some of the applications that demonstrate the usage of our client
are the following:

\begin{itemize}
    \item{Application \#1}
    \item{Application \#2}
    \item{Application \#3}
\end{itemize}

\pagebreak
